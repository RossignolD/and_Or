\documentclass[12pt]{article}
\usepackage[style=alphabetic]{biblatex}
\usepackage{bookmark}
\addbibresource{Mei_Rose_CONNOR.bib}
\usepackage[utf8]{inputenc}
\usepackage{graphicx}
\usepackage{epstopdf}
\usepackage{tipa}
\usepackage[margin=0.75in]{geometry}
\usepackage{enumitem}
\usepackage[center]{caption}
\usepackage{textcomp}
\usepackage{amsfonts}
\usepackage{amssymb}
\usepackage{amsmath}
\usepackage{amsthm}
\usepackage{mathrsfs}
\usepackage{float}
\usepackage{subcaption}
\usepackage{outlines}
\usepackage[table,xcdraw]{xcolor}
% \usepackage[dvipsnames]{xcolor}
\usepackage{hyperref}
\hypersetup{
    colorlinks=false,
    linkcolor=SeaGreen,
    urlcolor=SeaGreen,
    linktoc=section}
\hypersetup{final=true}

\renewcommand{\qedsymbol}{Q.E.D.}
\renewcommand{\phi}{\varphi}
\newcommand{\rr}{\ensuremath{\mathbb{R}}}
\newcommand{\zz}{\ensuremath{\mathbb{Z}}}
\newcommand{\qq}{\ensuremath{\mathbb{Q}}}
\newcommand{\pp}{\ensuremath{\mathbb{P}}}
\newcommand{\nn}{\ensuremath{\mathbb{N}}}
\newcommand{\cc}{\ensuremath{\mathbb{C}}}
\newcommand{\dd}{$\; \mathrm{d}$}
\newcommand{\lognot}{\ensuremath{\sim \!}}
\DeclareMathOperator{\forces}{\Vdash}
\DeclareMathOperator{\logbox}{\square}
\DeclareMathOperator{\logdiamond}{\lozenge}

\newtheorem{definition}{Definition}[section]
\newtheorem{theorem}{Theorem}[section]

\begin{document}

\title{When `And' is Too Much and `Or' is Not Enough}
\author{Mei Rose Connor}
\date{\today}

\maketitle

The source of inspiration for this project was the author's own research on the Sevenfold Predication of Jaina theology and cosmology. For background,
the Jaina take a pluralistic approach to theology and cosmology that is reflected in the Sanskrit word \emph{Anekantavada}, which glosses to something like `non-one-sidedness
thesis'. This pluralism comes into the Jaina system of reasoning known as \emph{Saptabhangivada}, or Sevenfold Predication Thesis \cite{Burch1964}. Employing the word \emph{syad}, each of the
seven truth predicates begins with a statement that can be translated as \emph{possibly}, \emph{it may be}, or \emph{in some ways}. The seven predicates are:

\begin{enumerate}[noitemsep]
    \item \emph{Syād-asti}, Possibly it exists.
    \item \emph{Syān-nāsti}, Possibly it does not exist.
    \item \emph{Syād-asti-nāsti}, Possibly it exists; possibly it does not exist.
    \item \emph{Syāt-asti-avaktavya\d{h}}, Possibly it exists; possibly it is not assertible.
    \item \emph{Syān-nāsti-avaktavya\d{h}}, Possibly it does not exist; possibly it is not assertible.
    \item \emph{Syād-asti-nāsti-avaktavya\d{h}}, Possibly it exists; possibly it does not exist; possibly it is not assertible.
    \item \emph{Syād-avaktavya\d{h}}, Possibly it is not assertible \cite{Ganeri2002}.
\end{enumerate}


The author chooses to abbreviate `it exists' as $\phi$.
It would seem, from these translations, that two truth values do not suffice to not introduce a contradiction or 
tautology of the form $\phi \; \wedge \lognot \phi$ or  $\phi \; \vee \lognot \phi$ for the third truth predicate.  This reveals the meaning of the title of the paper: somehow,
`and' is much too strong, causing a contradiction, but `or' is much too weak, causing a tautology. 

There is also the question of what to do with the seventh predicate, as well as those predicates involving it in combination with $\phi$ and $\lognot \phi$.
While it would seem possible to resolve it with a strong or weak Kleenean logical operator such as $U$, it is known that there are no tautologies in
a system with either type of Kleenean logical operator. \cite{Burns2023} It also leaves the question of the negation of these logical predicates unanswered, as the negation of the truth value $U$ is
$U$. The truth tables for the strong and weak Kleenean conjunction (\ref{KleeneStrongAnd} and \ref{KleeneWeakAnd} respectively), disjunction (\ref{KleeneStrongOr} and \ref{KleeneWeakOr} respectively), 
and negation (\ref{KleeneStrongNot} and \ref{KleeneWeakNot} respectively) are shown in the tables below for reference. The author does not deny, however,
that there are interesting and ``non--trivial'' \cite{Burns2023} consequence relations.

\begin{figure}[h!]
    \centering
    \begin{subfigure}{0.3\textwidth}
        \centering
            \begin{tabular}{ c | c | c | c }
                & $T$ & $U$ & $F$ \\
                \hline
                $T$ & $T$ & $U$ & $F$ \\
                \hline
                $U$ & $U$ & $U$ & $F$ \\
                \hline
                $F$ & $F$ & $F$ & $F$ \\
            \end{tabular}
            \caption{Kleenean strong conjunction}
            \label{KleeneStrongAnd}
    \end{subfigure}
    \begin{subfigure}{0.3\textwidth}
        \centering
            \begin{tabular}{ c | c | c | c }
                & $T$ & $U$ & $F$ \\
                \hline
                $T$ & $T$ & $T$ & $T$ \\
                \hline
                $U$ & $T$ & $U$ & $U$ \\
                \hline
                $F$ & $T$ & $U$ & $F$ \\
            \end{tabular}
            \caption{Kleenean strong \\ disjunction}
            \label{KleeneStrongOr}
    \end{subfigure}
    \begin{subfigure}{0.3\textwidth}
        \centering
            \begin{tabular}{ c | c }
                $T$ & $F$ \\
                \hline
                $U$ & $U$ \\
                \hline
                $F$ & $T$ \\
            \end{tabular}
            \caption{Kleenean strong negation}
            \label{KleeneStrongNot}
    \end{subfigure}
    \caption{The truth tables for Kleenean strong logical connectives}
\end{figure}

\begin{figure}[h!]
    \centering
    \begin{subfigure}{0.3\textwidth}
        \centering
            \begin{tabular}{ c | c | c | c }
                & $T$ & $U$ & $F$ \\
                \hline
                $T$ & $T$ & $U$ & $F$ \\
                \hline
                $U$ & $U$ & $U$ & $U$ \\
                \hline
                $F$ & $F$ & $U$ & $F$ \\
            \end{tabular}
            \caption{Kleenean weak conjunction}
            \label{KleeneWeakAnd}
    \end{subfigure}
    \begin{subfigure}{0.3\textwidth}
        \centering
            \begin{tabular}{ c | c | c | c }
                & $T$ & $U$ & $F$ \\
                \hline
                $T$ & $T$ & $U$ & $T$ \\
                \hline
                $U$ & $T$ & $U$ & $U$ \\
                \hline
                $F$ & $T$ & $U$ & $F$ \\
            \end{tabular}
            \caption{Kleenean weak \\ disjunction}
            \label{KleeneWeakOr}
    \end{subfigure}
    \begin{subfigure}{0.3\textwidth}
        \centering
            \begin{tabular}{ c | c }
                $T$ & $F$ \\
                \hline
                $U$ & $U$ \\
                \hline
                $F$ & $T$ \\
            \end{tabular}
            \caption{Kleenean weak negation}
            \label{KleeneWeakNot}
    \end{subfigure}
    \caption{The truth tables for Kleenean weak logical connectives}

\end{figure}
Adding a G\"{o}delian truth value $\mathbb{U}$ is also unsatisfactory. This is because the negation of $\mathbb{U}$ in a G\"{o}delian system is
$F$, which does not seem to match the interpretations of the connectives. For reference, the truth table of G\"{o}delian negation is given below. 

\begin{figure}[h!]
        \centering
            \begin{tabular}{ c | c }
                $T$ & $F$ \\
                \hline
                $\mathbb{U}$ & $F$ \\
                \hline
                $F$ & $T$ \\
            \end{tabular}
            \label{GoedelNot}
    \caption{The truth table for G\"{o}delian negation}
\end{figure}

Thus, the logic (and the linguistics and philosophy) seem to point the
author to a modal solution. Interpreting the box and diamond operators of modal logic in the provability sense (with $\logbox \phi$ being interpreted
as `$\phi$ is provable' and $\logdiamond \phi$ being interpreted as `$\phi$ does not introduce an inconsistency') while minding the caveat below,
seems to resolve some issues. However, one of the problems introduced in the beginning of this paper remains: how to prevent a semantically useless
statement like $\logdiamond \phi \; \vee \lognot \logdiamond \phi$. If one accepts the Law of Excluded Middle, this is a tautology. This result suggests
that the \emph{Saptabhangivada} may be non--classical not only in the structure of its truth values, but also in its notions of valitdity
and derivability. This is a work in progess, as the truth--value system of the \emph{Saptabhangivada} must be solidified before any notion of syntactic 
derivability or semantic validity can emerge. 

\bigskip

The paper will attempt to resolve some of the problems mentioned above, as well as make further inroads into the syntax and semantics of the 
\emph{Saptabhangivada}.

\printbibliography

\end{document}
